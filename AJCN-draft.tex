
%\documentclass[11pt,a4paper]{article}
\documentclass{article}
\usepackage{amsmath, amsthm, amsfonts}
%\usepackage[utf8]{inputenc}
\usepackage[english]{babel}
\usepackage{caption}
\usepackage{subcaption}
\usepackage{subfig}
\usepackage{setspace}
\usepackage{float}
%\usepackage{subfigure}
\usepackage{caption}
\usepackage{subcaption}
\usepackage{graphicx}% Include figure files
\usepackage{dcolumn}% Align table columns on decimal point
\usepackage{bm}% bold math
\usepackage{multicol}
\usepackage[margin=2.5cm]{geometry}
\graphicspath{{/Users/gowinsja/desktop/images_for_tex/}} 
%\usepackage{bigfoot}
%\usepackage{ftnright}

\begin{document}


\noindent
\LARGE{\textbf{Dynamic Mathematical Model of Infant Energy Imbalance,  Body Composition Change, and Growth from Birth to 18 Years Old}}\footnote{From the Laboratory of Biological Modeling, National Institute of Diabetes, Digestive, and Kidney Diseases, National Institute of Health, Bethesda, MD}\footnote{Supported by the Intramural Program of the NIH, NIDDK}\footnote{Reprints not available, Address correspondence to}\\

\noindent \large{\it{Aaron Gowins, Danielle Toupo, Carson Chow, and Kevin Hall}}
\begin{multicols}{2}


\begin{small}
\section*{ABSTRACT}
\textbf{Background:} Addressing the increasing prominence of childhood obesity will require an understanding of the underlying physiology and physical constraints. Clinicians and policy makers can benefit from the ability to compare the changes in body composition and energy use associated with healthy childhood growth to those associated with obesity.  \\
\textbf{Objective:} We sought to develop a single model of the relationships between the intake and expenditure of energy and the deposition of tissue throughout childhood growth for both healthy weight and overweight subjects.\\
\textbf{Design:} By relating the changes in energy content in various tissues of children experiencing healthy growth to their nutrient intakes, we developed a computational simulation of the energy requirements and tissue deposition of healthy growth. We then extended this strategy to encompass the development of obesity.\\
\textbf{Results:} Our model accurately reproduces data from a variety of studies of healthy and overfed subjects other than those studies used to calibrate the model. After calibration, our model makes no reference to the calibration data, and relies only on the model parameters, the inputs of energy intake and expenditure, and age to produce simulations. We found that not only are overweight children taller on average than their peers, but that they are also shorter than their peers for their fat free mass value. We found that a single parameterized growth function of time is not sufficient to capture the body composition dynamics of the development of obesity for a model that begins at birth. \\
\textbf{Conclusions:} The model we present describes the typical deposition of fat and fat free tissue, as well as the growth of major organs, as a function of energy intake and expenditure dynamically throughout growth. Our model also predicts the distribution of tissue during periods of time where energy absorption differs from reference, as well as predicting the energy that must have been absorbed in order to reach a growth trajectory that differs from reference.


\section*{KEY WORDS} Infant growth, childhood growth, mathematical model, body composition, energy requirements, childhood obesity
\section*{INTRODUCTION}
Despite widespread efforts to increase awareness and implement policy, the obesity epidemic continues to be a crisis in developed countries around the world. As obesity rates climb, an alarming number of children are being affected, and at ever younger ages.\cite{p29}. However, studying the complex physiology that underlies childhood growth poses exceptional challenges. Measuring energy intake requires continuous monitoring of the energy content of the child's diet. The most precise methods of measuring energy expenditure, respiration calorimetry and the doubly-labeled water (DLW) method, are costly and require specialized laboratories. Accurate measurements of body composition of infants is especially rare, due primarily to the resistance of clinicians and patients' guardians to expose infants and small children to radiation during Dual-energy-X-ray-absorptiometry (DXA) measurements. Each of these methods introduce error, have prohibitive costs, and many requirements simply cannot be applied to infants in an appropriate and accurate manner. Our aim was to provide a method to produce estimates of the most difficult to measure and most important factors leading to childhood obesity using easier to obtain and less expensive measurements. Here we extend the well-validated model of \cite{p1}, by including ages from birth, introducing a number of revisions, and representing substantial additional physiology. Evidence indicates that body composition even prior to birth is associated with later body composition and obesity \cite{p30}, highlighting the importance of understanding body composition changes and energy use in early growth. We present a mechanistic model based on established physiology and the physical properties of tissue and energy, and provide quantitative insights into the body composition changes related to varying energy intake and expenditure. 
\section*{METHODS}
Our mathematical model of energy partitioning between fat free mass (FFM) and fat mass (FM) from birth to adulthood is an extension of the previously published child and adult models \cite{p1}, and is presented in detail in the appendix. We begin with the fundamental observation that energy is in balance, and that changes in mass are governed by the ratio between energy absorbed through consumption and energy expended through maintenance and activity. The model is designed to predict the distribution of accumulated tissue between FFM and FM as a function of age, energy imbalance, and current body composition. 
Gender-specific trajectories for FFM and FM were fitted using age or mass dependent functions drawn from previously collected data on relevant physiological processes. Through combining these various independent functions, we constructed a model that substantially reproduces all calibration data and makes reasonable predictions of validation data, while providing insights into the physiological processes that underlie the dynamics of childhood growth. 
\subsection*{Healthy Weight Model Calibration}
In keeping with successful prior approaches, we developed an ordinary differential equation that we calibrated with data as shown in each figure. We began by reproducing the changing proportion of FFM and FM from \cite{p8,p9}, as well as the energy required for both maintenance and deposition of mass. In order to establish the physiological mechanisms associated with this energy use and mass partitioning, we included a number of processes that occur as infants mature. These include the increasing energy density of FFM, the increase in cellularity, the growth of major organs and consequent changes in metabolic requirements, the thermic effect of feeding, adaptive thermogenesis, changes in physical activity, and changes in the glycolytic rate of the brain as it grows and matures. \\
\bigskip

A \hspace{3.4cm} B \\
\includegraphics[width=4cm,height=3cm]{TCfinal} 
%\begin{footnotesize}
%Figure 2. Calculation of male Fomon/Haschke healthy reference Energy Imbalance Gap as a function of age \cite{p8,p9}.
%\end{footnotesize}
\includegraphics[width=4cm,height=3cm]{Efinal} \\
\begin{footnotesize}
Figure 1. Model simulation of body composition for healthy weight male subjects (A) shown with body composition data from \cite{p8,p9,p36} and Total Energy Expenditure (B) from \cite{p36,p31}
\end{footnotesize}\\
\bigskip
\bigskip

A \hspace{3.4cm} B \\
\includegraphics[width=4cm,height=3cm]{EIGfinal} 
%\begin{footnotesize}
%Figure 2. Calculation of male Fomon/Haschke healthy reference Energy Imbalance Gap as a function of age \cite{p8,p9}.
%\end{footnotesize}
\includegraphics[width=4cm,height=3cm]{GrowthFunctionFinal} \\
\begin{footnotesize}
Figure 2. Energy Imbalance Gap as a function of age (A) and childhood growth represented as increased lean tissue deposition above adult Forbes partitioning ratio \cite{p10} (B) for males. \\
\end{footnotesize} \\


The model is developed in Matlab\textsuperscript{\textregistered} Version 8.4.0.150421(R2014b), using the Curve Fitting Toolbox Version 3.5(R2014b), and employs the ODE45 Runge-Kutta algorithm to numerically integrate. Some data uploading, subsetting, and exploration were done in R. The integrator function and all other codes are available in the appendix, in addition to a full description of model development. 
\subsection*{Healthy Weight Model Validations}
We validated the healthy weight model using a number of datasets not used for calibration. We included a number of studies of infants in addition to older subjects, including studies that were previously used in \cite{p1}, as well as others. Only energy intake values and initial conditions were altered to simulate validation data; birth weights were adjusted where they were included in the study and presumed to be identical to the calibration set where missing. No other model parameters were adjusted during the validations. \\
A \hspace{3.4cm} B \\
\includegraphics[width=4cm,height=3cm]{SeminarWellsH} 
%\begin{footnotesize}
%Figure 2. Calculation of male Fomon/Haschke healthy reference Energy Imbalance Gap as a function of age \cite{p8,p9}.
%\end{footnotesize}
\includegraphics[width=4cm,height=3cm]{SeminarS} \\
\begin{footnotesize}
Figure 2. Energy Imbalance Gap as a function of age (A) and childhood growth represented as increased lean tissue deposition above adult Forbes partitioning ratio \cite{p10} (B) for males. \\
\end{footnotesize} \\



\includegraphics[width=8.5cm,height=6cm]{SeminarWellsH}
\begin{footnotesize}
Figure 4. Healthy male subjects model validation from \cite{p7}
\end{footnotesize} \\
\includegraphics[width=8.5cm,height=6cm]{SeminarS}
\begin{footnotesize}
Figure 5. Healthy female subjects model validation from \cite{p37}. 
\end{footnotesize} 
\subsection*{Overweight Model Calibration}
We supplied the capability to model overfeeding by simulating an increase in energy intake and calibrating to \cite{p7} for overfed subjects. 
\includegraphics[width=8.5cm,height=6cm]{SeminarWellsTC}
\begin{footnotesize}
Figure 6. Overweight male subjects model calibration from \cite{p7}. \\
\end{footnotesize} \\
\subsection*{Overweight Model Validation}
We then performed validation simulations similar to above, by simulating data not used in model calibration, and adjusting only initial conditions and energy intake values for validation.
\includegraphics[width=8.5cm,height=6cm]{SeminarGTC}
\begin{footnotesize}
Figure 7. Overweight male subjects model validation from \cite{p37}. \\
\end{footnotesize} \\
\subsection*{Energy Intake and Expenditure}

\section*{RESULTS}
\subsection*{Defining Healthy Growth Trajectories}
\subsection*{Simulating Body Compositions in Obesity}
\subsection*{Simulating Energy Absorption, Intake, and Expenditure}
\section*{DISCUSSION}
\section*{REFERENCES}
\newpage
\end{small}
\begin{thebibliography}{99}

\tiny{

\bibitem[Hall et.al., 2013]{p1} Hall KD, Butte NF, Swinburn BA, Chow CC (2013)
\newblock Dynamics of childhood growth and obesity: development and validation of a quantitative mathematical model.
\newblock \emph{Lancet Diabetes and Endocrinology} 12(3), http://dx.doi.org/10.1016/S2213-8587(13)70051-2
\bibitem[Stunkard et. al., 1999]{p4} Stunkard AJ, Berkowitz, RI, Stallings VA, Scholler, DA (1999)
\newblock Energy intake, not energy output, is a determinant of body size in infants.
\newblock \emph{Am J Clin Nutr} 1999;69:524-30
\bibitem[Rahmandad, 2014]{p2} Rahmandad H (2014)
\newblock Human Growth and Body Weight Dynamics: An Integrative Systems Model.
\newblock \emph{PLoS ONE} 9(12): e114609. doi:10.1371/journal.pone.0114609
\bibitem[Tennefors et. al., 2003]{p3} Tennefors C, Coward WA, Hernell O, Wright A, Forsum E (2003)
\newblock Total energy expenditure and physical activity level in healthy young Swedish children
9 or 14 months of age.
\newblock \emph{European Journal of Clinical Nutrition} 57, 647-653 0954-3007/03
\bibitem[Hall et.al., 2011]{p5} Hall KD, Saks G, Chow CC, Chandramohan D, Wang YC, Gortmaker SL, Swinburn BA (2011)
\newblock Quantification of the effect of energy imbalance on bodyweight.
\newblock \emph{Lancet} 378: 826-37
\bibitem[Salbe et. al., 2002]{p6} Salbe AD, Weyer C, Harper I, Lindsay RS, Ravussin E, et. al. (2002)
\newblock Assessing risk factors for obesity between childhood and adolescence: II. Energy metabolism and physical activity.
\newblock \emph{Pediatrics} 110: 307-314.




%------------------------------------------------

%------------------------------------------------


\bibitem[Wells et.al., 2006]{p7} Wells JC, Fewtrell MS, Williams JE, Haroun D, Lawson MS, Cole TJ (2006)
\newblock Body composition in normal weight, overweight and obese children: matched case-control analyses of total and regional tissue masses, and body composition trends in relation to relative weight.
\newblock \emph{Int J Obes (Lond)} 30: 1506-13.
\bibitem[Fomon et.al., 1982]{p8} Fomon SJ, Haschke F, Zeigler EE, Nelson SE (1982)
\newblock Body composition of reference children from birth to age 10 years.
\newblock \emph{Am J Clin Nutr} 35(5 Suppl): 1169-75.
\bibitem[Haschke, 1989]{p9} Haschke F (1989)
\newblock Body composition during adolescence. In: Klish WJ, Kretchmer N, eds. Body composition measurements in infants and children. Columbus, OH: Ross Laboratories; 76-83.
\bibitem[Forbes, 1987]{p10} Forbes GB (1987)
\newblock Lean body mass-body fat interrelationship in humans.
\newblock \emph{Nutr Rev} 45(8): 225-31.
\bibitem[Gutin et. al., 2002]{p11} Gutin B, Barbeau P, Owens S, Lemmon CR, Bauman M, Allison J, Kang H-S, Litaker MS (2002)
\newblock Effects of exercise intensity on cardiovascular fitness, total body composition, and visceral adiposity of obese adolescents.
\newblock \emph{Am J Clin Nutr} 75: 818-26.
\bibitem[Chomtho et. al., 2008]{p12} Chomtho S, Wells JCK, Williams JE, Davies PSW, Lucas A, Fewtrell MS (2008)
\newblock Infant growth and later body composition: evidence from the 4-compartment model.
\newblock \emph{Am J Clin Nutr} 87: 1776-84.








%------------------------------------------------



\bibitem[Ellis et.al., 2000]{p13} Ellis KJ, Shypailo RJ, Abrams SA, Wong WW (2000)
\newblock The reference child and adolescent models of body composition - A contemporary comparison.
\newblock \emph{In Vivo Body Composition Studies} 904: 374-382.
\bibitem[Gutin et.al., 1996]{p14} Gutin B, Litaker M, Islam S, Mancos T, Smith C, Treiber F (1996)
\newblock Body-composition measurement in 9-11-y-old children by dual-energy X-ray absorptiometry, skinfol-thicckness measurements, and bioimpedance analysis.
\newblock \emph{Am J Clin Nutr} 63: 287-92
\bibitem[NHANES 2003-2004]{p15} CDC NHANES
\newblock Centers for Disease Control and Prevention (CDC). National Center for Health Statistics (NCHS). National Health and Nutrition Examination Survey Data. Hyattsville, MD: U.S. Department of Health and Human Services, Centers for Disease Control and Prevention, [2003-2004][http://wwwn.cdc.gov/nchs/nhanes/search/nhanes03-04.aspx].
\bibitem[Rogers et.al., 2006]{p16} Rogers IS, Ness AR, Steer CD, Wells, JCK, Emmett PM , Reilly JR, Tobias J, Smith GD (2006)
\newblock Associations of sixe at birth and dual-energy X-ray absortiometry measures of lean and fat mass at 9 to 10 y of age.
\newblock \emph{Am J Clin Nutr} 84:739-47.
\bibitem[Carberry et.al., 2010]{p17} Carbery AE, Colditz PB, Lingwood BE (2010)
\newblock Body composition from birth to 4.5 months in infants born to non-obese women.
\newblock \emph{Pediatric Research} Vol. 68 No. 1 0031-3998/10/6801-0084.
\bibitem[Chomtho et.al., 2008]{p18} Chomtho S, Wells JCK, Williams JE, Lucas A, Fewtrell MS (2008)
\newblock Associations between birth weight and later body composition: evidence from the 4-component model
\newblock \emph{Am J Clin Nutr} 88:1040-8.







%------------------------------------------------




\bibitem[de Bruin et.al., 1998]{p19} de Bruin NC, Degenhart J, Gal S, Westerterp KR, Stijnen T, Visser HKA (1998)
\newblock Energy utilization and growth in breast-fed and formula-fed infants measured prospectively during the first year of life.
\newblock \emph{Am J Clin Nutr} 67:885-96.
\bibitem[Dulloo et.al., 1993]{p20} Dulloo AG, Girardier L (1993)
\newblock Adaptive role of energy expenditure in modulating body fat and protein deposition catch-up growth after early undernutrition.
\newblock \emph{Am J Clin Nutr} 58:614-621.
\bibitem[Martins et.al., 2004]{p21} Martins PA, Hoffman DJ, Fernandez MTB, Nascimento CR, Roberts SB, Sesso R, Sawaya AL, (2004)
\newblock Stunted children gain less lean body mass and more fat mass than their non-stunted counterparts: a prospective study.
\newblock \emph{British J Nutr} 92:819-825.
\bibitem[Clemente et.al., 2011]{p22} Clemente AP, Santos CD, Martins VJ, Benedito-Silva AA, Albuquerque MP, Sawaya AL, (2011)
\newblock Mild stunting is associated with higher body fat: study of a low-income population.
\newblock \emph{J Pediatr} 87(2):138-144.
\bibitem[Radhakrishna et.al., 2010]{p23} Radhakrishna KV, Kulkarni B, Balakrishna N, Rajkumar H, Omkar C, Shatrugna V, (2010)
\newblock Composition of weight gain during nutrition rehabilitation of severely under nourished children in a hospital based study from India.
\newblock \emph{Asia Pac J Clin Nutr} 19(1):8-13.
\bibitem[Chow, Hall, 2008]{p24} Chow CC, Hall KD, (2008)
\newblock The dynamics of human body weight change.
\newblock \emph{PLoS computational biology} 4(3):e1000045.
\bibitem[Hall, 2007]{p25} Hall, (2007)
\newblock Body fat and fat-free mass inter-relationships: Forbes's theory revisited. 
\newblock \emph{Br J Nutr} 97(6): 1059-63.
\bibitem[Elia, 1992]{p26} Elia M (1992)
\newblock Organ and tissue contribution to metabolic rate.. In: Kinney JM, Tucker HN, eds. Energy Metabolism: Tissue Determinants and Cellular Corollaries. New York: Raven Press; 1992: 61-79.
\bibitem[Altman, Dittmer, 1962]{p27} Altman PL, Dittmer DS (1962)
\newblock Growth, including reproduction and morphological development. Washington DC: Federation of American Societies for Experimental Biology.
\bibitem[Garn et. al., 1975]{p28} Garn SM, Clark DC, Lowe CU, Forbes G, Owen GM, Smith NJ, Weil WBJ, Nichaman MZ, Johansen E, Rowe N (1975)
\newblock Nutrition, Growth, Development, and Maturation: Findings From the Ten-State Nutrition Survey of 1968-1970 Ad Hoc Committee To Review the Ten-State Nutrition Survey
\newblock \emph{PEDIATRICS} 56(2):306-319.
\bibitem[Druet et. al., 2011]{p29} Druet C, Stettler N, Sharp S, Simmons RK, Cooper C, Smith GD, Ekelund U, L�vy-Marchal C, Jarvelin M-R, Kuh D, Ong KK (2011)
\newblock Prediction of childhood obesity by infancy weight gain: an individual-level meta-analysis.
\newblock \emph{Paediatric and Perinatal Epidemiology} 26, 19?26.
\bibitem[McMillen et. al., 2009]{p30} McMillen CI, Rattanatray L, Duffield JA, Morrison JL, MacLaughlin SM, Gentili S, Muhlhausler BS (2009)
\newblock The Early Origins of Later Obesity: Pathways and Mechanisms
\newblock Early Nutrition Programming and Health Outcomes in Later Life
Volume 646 of the series Advances in Experimental Medicine and Biology, Springer, pp 71-81

\bibitem[Butte, 2005]{p31} Butte NF (2005)
\newblock Energy requirements of infants.
\newblock \emph{Pub Health Nutr} 8(7A), 953-967.
\bibitem[Butte, 2006]{p32} Butte NF (2006)
\newblock Energy requirements of infants and children.
\newblock \emph{} .
\bibitem[Butte, 2000]{p33} Butte NF, Wong WW, Hopkinson JM, et. al. (2000)
\newblock Body composition during the first two years of life: an updated reference.
\newblock \emph{Pediatr Res} 47:578-585.
\bibitem[Aleksy et. al., 1998]{p34} Aleksy U, Kersting M, Sichert-Hellert W, Manz F,  Schoch G. (1998)
\newblock Energy Intake and Growth of 3-to-36-Month-OLD German Infants and Children.
\newblock \emph{Ann Nutr Metab} 42:68-74.
\bibitem[Arsenault et. al., 2008]{p35} Arsenault JE, Lopez de Romana D, Penny ME, Van Loan MD,  Brown KH. (2008)
\newblock Additional Zinc Delivered in a Liquid Supplement, but Not in a Fortified Porridge, Increased Fat-Free Mass Accrual among Young Peruvian Children with Mild-to-Moderate Stunting.
\newblock \emph{J Nutr} 138:108-114.
\bibitem[Torun, 2005]{p36} Torun B (2005)
\newblock Energy requirements of children and adolescents
\newblock \emph{Pub Health Nutr} 8(7A),968-993
\bibitem[Lazzer et. al., 2005]{p37} Lazzer S, Boirie Y, Poissonnier C, Petit I, Duche P, Tallardat M, Meyer M, Vermorel M (2005)
\newblock Longitudinal changes in activity patterns, physical capacities, energy expenditure, and body composition in severely obese adolescents during a multidisciplinary weight-reduction program.
\newblock \emph{Int J Obes} 29,37-46
\bibitem[Spadano et. al., 2005]{p37} Spadano JL, Bandini LG, Must A, Dallal GE, Dietz WH (2005)
\newblock Longitudinal changes in energy expenditure in girls from late childhood through mid adolescence.
\newblock \emph{Am J Clin Nutr} 81,1102-09
}
\end{thebibliography}
\section*{APPENDIX}
\end{multicols}
 
\end{document}

%\documentclass{article}
%\usepackage[utf8]{inputenc}
%\usepackage[english]{babel}
% 
%\usepackage{multicol}
% 
%\begin{document}
%\begin{multicols}{3}
%[
%\section*{First Section}
%All human things are subject to decay. And when fate summons, Monarchs must obey.
%]
%Hello, here is some text without a meaning.  This text should show what 
%a printed text will look like at this place.
%If you read this text, you will get no information.  Really?  Is there 
%no information?  Is there...
%\section*{ABSTRACT}
%\section*{KEY WORDS}
%\section*{INTRODUCTION}
%\section*{METHODS}
%\section*{RESULTS}
%\section*{DISCUSSION}
%\section*{REFERENCES}
%\end{multicols}
% 
%\end{document}
